
\begin{table}
\begin{center}
\begin{tabular}{l c c c c c c}
\hline
 & Bw ±1 & Bw ±2 & Bw ±3 & Bw ±4 & Bw ±5 & Bw ±6 \\
\hline
salient\_voting & $-0.07$          & $-0.02$          & $-0.02$          & $-0.03$          & $-0.04$          & $-0.02$          \\
                & $ [-0.21; 0.06]$ & $ [-0.11; 0.06]$ & $ [-0.09; 0.04]$ & $ [-0.09; 0.02]$ & $ [-0.09; 0.01]$ & $ [-0.07; 0.02]$ \\
\hline
R$^2$           & $0.09$           & $0.07$           & $0.07$           & $0.07$           & $0.07$           & $0.07$           \\
Adj. R$^2$      & $0.07$           & $0.06$           & $0.06$           & $0.06$           & $0.06$           & $0.06$           \\
Num. obs.       & $1480$           & $2432$           & $3363$           & $4240$           & $4829$           & $5402$           \\
RMSE            & $0.58$           & $0.59$           & $0.60$           & $0.60$           & $0.60$           & $0.60$           \\
\hline
\multicolumn{7}{l}{\scriptsize{$^*$ Null hypothesis value outside the confidence interval.}}
\end{tabular}
\caption{Bandwidth Sensitivity Analysis: Effect of Salient Voting on Party Identification. The table shows the estimated effect of salient voting on party identification across different bandwidth specifications in the fuzzy regression discontinuity design. Each column represents a two-stage least squares estimate using voting eligibility as an instrument for salient voting, with country fixed effects included. Standard errors are robust (HC1) and 95\% confidence intervals are shown in brackets.}
\label{table:coefficients_pid}
\end{center}
\end{table}
