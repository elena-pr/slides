
\begin{table}
\begin{center}
\begin{tabular}{l c c}
\hline
 & ITT Effect & 2SLS Optimal \\
\hline
(Intercept)         & $1.88^{*}$       &                 \\
                    & $ [ 1.83; 1.92]$ &                 \\
voting\_eligibility & $-0.01$          &                 \\
                    & $ [-0.08; 0.06]$ &                 \\
year\_cen           & $0.01^{*}$       &                 \\
                    & $ [ 0.00; 0.02]$ &                 \\
salient\_voting     &                  & $0.19^{*}$      \\
                    &                  & $ [0.10; 0.28]$ \\
\hline
R$^2$               & $0.00$           & $0.19$          \\
Adj. R$^2$          & $0.00$           & $0.18$          \\
Num. obs.           & $18368$          & $15825$         \\
RMSE                & $1.35$           & $1.22$          \\
\hline
\multicolumn{3}{l}{\scriptsize{$^*$ Null hypothesis value outside the confidence interval.}}
\end{tabular}
\caption{Effect of Voting Eligibility and Salient Voting on Weighted Affective Polarisation. The table presents results from a fuzzy regression discontinuity design examining how voting eligibility and participation in salient elections affect population-weighted affective polarisation. Column 1 shows the intent-to-treat (ITT) effect of voting eligibility using OLS with robust standard errors. Column 2 presents the two-stage least squares (2SLS) estimate using voting eligibility as an instrument for salient voting, estimated at the optimal bandwidth. The dependent variable is affective polarisation weighted by population size and measured on a continuous scale. Standard errors are robust (HC1) and 95\% confidence intervals are shown in brackets. The models include country fixed effects}
\label{table:main_results_w}
\end{center}
\end{table}
