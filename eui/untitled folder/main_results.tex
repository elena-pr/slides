
\begin{table}
\begin{center}
\begin{tabular}{l c c}
\hline
 & ITT Effect & 2SLS Optimal \\
\hline
(Intercept)         & $2.17^{*}$       &                 \\
                    & $ [ 2.13; 2.22]$ &                 \\
voting\_eligibility & $0.02$           &                 \\
                    & $ [-0.05; 0.09]$ &                 \\
year\_cen           & $0.01$           &                 \\
                    & $ [-0.00; 0.02]$ &                 \\
salient\_voting     &                  & $0.19^{*}$      \\
                    &                  & $ [0.10; 0.29]$ \\
\hline
R$^2$               & $0.00$           & $0.10$          \\
Adj. R$^2$          & $0.00$           & $0.10$          \\
Num. obs.           & $16073$          & $12498$         \\
RMSE                & $1.27$           & $1.22$          \\
\hline
\multicolumn{3}{l}{\scriptsize{$^*$ Null hypothesis value outside the confidence interval.}}
\end{tabular}
\caption{Effect of Voting Eligibility and Salient Voting on Affective Polarisation. The table presents results from a fuzzy regression discontinuity design examining how voting eligibility and participation in salient elections affect affective polarisation. Column 1 shows the intent-to-treat (ITT) effect of voting eligibility using OLS with country fixed effects. Column 2 presents the two-stage least squares (2SLS) estimate using voting eligibility as an instrument for salient voting, estimated at the optimal bandwidth. The dependent variable is affective polarisation measured on a continuous scale. Standard errors are clustered at the country level and 95\% confidence intervals are shown in brackets. The models include country fixed effects}
\label{table:main_results}
\end{center}
\end{table}
